\documentclass[10pt,letterpaper]{article}
\usepackage[utf8]{inputenc}
\usepackage[spanish]{babel}
\usepackage{amsmath}
\usepackage{amsfonts}
\usepackage{amssymb}
\usepackage{graphicx}
\usepackage[left=2cm,right=2cm,top=2cm,bottom=2cm]{geometry}
\author{Hernan Casanova, Javier Perez, Arturo Reyes y Leon Vidal}
\title{Transmision y recepcion inalambrica con Arduino Uno}
\date{}
\usepackage{multicol}
\pagestyle{empty}

\begin{document}

\maketitle
\thispagestyle{empty}
\begin{multicols}{2}
\begin{abstract} 

La comunicación inalámbrica, es hoy en día, el área de investigación científica con mas proyectos en desarrollo y proyección a futuro, sobre todo en la industria de las Telecomunicaciones. Es por esto, que como estudiantes de Ingeniería Civil Informática, nos vimos desafiados a programar dos dispositivos Arduino para que logren comunicarse y transmitir un mensaje básico de forma inalámbrica. Para esto fue necesario incorporar un transmisor y un receptor a los dispositivos Arduino, también utilizamos una librería especializada en comunicación inalámbrica para Arduino, llamada VirtualWire. Nuestra misión fue desarrollar un algoritmo para poder determinar que calidad de señal se logra transmitir a distintas distancias entre nuestros dispositivos.

El objetivo de este informe es mostrar al lector una descripción detallada de la secuencia de la investigación, actividades y pruebas realizadas para lograr la comunicación inalámbrica.


El resumen (más conocido como abstract, en inglés) aparece inmediatamente después del título del artículo. Presenta:

    el contexto del estudio
En la realización de la segunda tarea de Comunicaciones    
    
    el propósito del estudio
    los procedimientos básicos (selección de sujetos del estudio o animales de laboratorio, los métodos observacionales o analíticos)
    los descubrimientos principales (dando tamaños específicos de los efectos y su importancia estadística, si fuera posible)
    las conclusiones principales.
    palabras claves

Debe enfatizar los aspectos nuevos o importantes del estudio o de las observaciones.

En sitios de búsqueda (como PubMed) o en revistas con licencias, el resumen es lo único que se muestra de un artículo científico.




\end{abstract}

\section{Introducción}
La introducción presenta el tema a tratar en el artículo y suele responder a la pregunta del por qué se ha realizado el estudio; debe contener la hipótesis que se intenta demostrar mediante el estudio o experimento realizado. Suele no tener más de dos párrafos y a veces incluye un compendio de las últimas averiguaciones en el tema.

Características generales

    Permite la flexibilidad y variedad temática, de igual manera parte de un análisis descriptivo sobre temas históricos, teóricos, científicos, políticos, culturales, económicos y sociales de actualidad.
    Se insinúa que la redacción del artículo debe partir de hechos concretos y no de reflexiones de tipo filosófico o consideraciones generales del asunto a tratar.
    Se trabaja y analiza directamente sobre hechos que no tienen otra finalidad rigurosa que la de informar sin tener que trasmitir datos puntuales. A través de la persuasión y la seducción, el articulista describe los acontecimientos más o menos actuales.
    
    ejemplo de una formula
    \[c=\frac{a+b}{c_1}\]


\section{Métodos y materiales}
La sección de métodos sólo debe incluir la información que estaba disponible en el momento en que se escribió el plan o protocolo del estudio. Cualquier información que se consiguió durante el estudio debe consignarse en la sección de Resultados. Usualmente los métodos describen técnicas o métodos existentes haciendo énfasis en como se aplicarán al estudio concreto del artículo científico. La parte de materiales describe las muestras u objetos de estudios, su descripción, su procedencia y sus características generales relevantes para el estudio.
\subsection{Métodos}
Aqui se describen los metodos utilizados para resolver la problematica.

\subsection{Materiales}
Aqui se especifican los materiales utilizados.

\section{Resultados}
Donde se presentan los resultados obtenidos, en estudios experimentales o simulaciones computaciones se suelen acompañar con tablas o gráficos que resumen aspectos cuantitativos y cualitativos de los nuevos resultados obtenidos en el estudio.

\section{Discusión}
En la discusión se retoman los resultados obtenidos y se comparan con otros previos, se contextualiza su importancia, así como las implicaciones prácticas y teóricas de los mismos. En esta sección se mencionan investigaciones futuras, así como posibles usos de los resultados. En esta parte frecuentemente se tienen en cuenta posibles objeciones, limitaciones y comentarios de los resultados.

\section{Conclusiones}

\section{Agradecimientos}

\section{Bibliografía}
Una enumeración de la bibliografía consultada y citada. Mayormente esta bibliografía consta de otros artículos científicos, usualmente recientes, y sólo muy ocasionalmente se citan artículos antiguos que fueron históricamente importantes o seminales y libros con resultados generalistas. Existen diversos sistemas de cita de artículos, usualmente diversos campos científicos usan su propio estilo de citación.

\begin{thebibliography}{99}
\bibitem{}{primer libro kdjksd jhkjh}
\end{thebibliography}

al final van las fotos de todos los autores.
\includegraphics[width=3cm]{DSC_2600.JPG} 


\end{multicols}
\end{document}